\section{Задание 1. Интеграл функции одной переменной}

\textbf{Условие.}

В задачах проведите исследование:

1.Составьте математическую модель задачи: введите обозначения, выпишите данные,  составьте уравнение (систему уравнений), содержащее неизвестное.

2.Решите задачу аналитически.

3.Сделайте графическую иллюстрацию к решению задачи.

4.Запишите ответ.

\vspace{5mm}

\begin{multicols}{2}
    Вычислите силу давления воды на пластинку,
    вертикально погруженную в воду,
    считая, что удельный вес воды равен 9.81 кН/м$^3$.
    Результат округлите до целого числа.
    Форма, размеры и расположение пластины указаны на рисунке.

    \includegraphics[width=5cm]{images/1a1}

\end{multicols}

\vspace{10mm}

\textbf{Решение.}

\vspace{5mm}

\begin{minipage}{\linewidth}

    \begin{wrapfigure}{r}{0pt}
        \includegraphics[height=60mm]{images/1a2}
    \end{wrapfigure}

    Привяжем начало декартовой системы координат к верхней точке ромба на картинке. Ось $Oy$ в положительном направлении направим вниз.

    Сила давления воды вычисляется по формуле: $F = p(h) \cdot S, p(h) = \rho g h = \gamma h$, где $\gamma = $9.81 кН/м$^3$, $S$ - площадь участка.

    Разделим ромб на две части: верхнюю и нижнюю.

    Сделаем равное дробление для каждой части горизонтальным линиями $y_0, y_1, y_2, \dots, y_n$ -- площадь получившихся трапеций вычисляется по формуле $f(\xi_i) \Delta y_i$,
    где $\Delta y_i = y_i - y_{i - 1}$, $\displaystyle \xi_i = \frac{y_i + y_{i - 1}}{2}$ - ордината средней линии трапеции, $f(y)$ - длина разреза ромба на уровне $y$.

    Для верхней половины ромба $f_1(y) = y$, для нижней $f_2(y) = 4 - y$

    Получим предел суммы $\displaystyle \lim_{\substack{n \to \infty \\ \tau \to 0}} \sum_{i = 1}^{n} \Delta y_i f(\xi_i) p(\xi_i)$ или же интеграл
    $\displaystyle \int_a^b f(y) p(y) dy$

    Сила давления $\displaystyle F = \int_0^2 f_1(y) p(y) dy + \int_2^4 f_2(y) p(y) dy = \gamma \int_0^2 y^2 dy + \gamma \int_2^4 (4y - y^2) dy =
    \gamma (\frac{y^3}{3} \Big|_0^2 + (2y^2 - \frac{y^3}{3}) \Big|_2^4) = \gamma (\frac{8}{3} + 32 - \frac{64}{3} - 8 + \frac{8}{3}) = 8 \gamma = 78,48$ кН

    Еще эту задачу можно решать так: $\displaystyle \int_0^2dA = \int_0^2p(y)dl = \int_0^2p(y)f_1(y)dy$, где $f_1(y)$ - какой длины отрезок у точек
    с координатами y от текшего значенния принимает значение $p(y)$, или иначе какой прямой ограничен наш ромб на текущем уровне.  
    Такие же рассуждения производятся со второй правильной частью интегрирования ромба и по расчетам выше и происходят вычисления

\end{minipage}

\vspace{14mm}

\textit{Ответ}: $78,48$ кН
\clearpage
